\documentclass{article}
\usepackage[utf8]{inputenc}
\usepackage[english,russian]{babel}
\usepackage{amssymb,amsfonts,amsmath,mathtext}
\usepackage{enumerate,indentfirst,tikz,pgf}


% ----------- Дополнительные спецификации ----------
\newcounter{tasknum}
\newcommand{\task}{\addtocounter{tasknum}{1}
\textbf{Задача \arabic{tasknum}.\,\,}}
%---------------------------------------------------


\begin{document}
\section*{Задания к лекции 1. Базовые понятия теории вероятностей.}

\subsection*{Вариант 1}


\task Из полного набора 28 костей домино наудачу берутся 5 костей.
Найти вероятность того, что среди них будет хотя бы одна кость с шестью очками.

\task Монета подбрасывается 19 раз. Найти вероятность того, чточисло появлений герба четно.

\task В $n$ ящиках размещают $3n$ шаров. Найти вероятность того,
что ни один ящик не пуст.

\task В круг вписан равносторонний треугольник. Точка наудачу бросается в круг. Найти вероятность того, 
что она попадет в треугольник.

\task В урне 7 белых и 3 черных шара. 
Без возвращения извлекаются 3 шара. Известно, что среди них есть черный шар. 
Какова вероятность того, что другие два шара белые?

\task Случайная величина определяется исходом подбрасывания монеты. $0$ -- выпадение <<решки>>,
$1$ -- выпадение <<орла>>. Определить математическое ожидание и дисперсию этой случайной величины, если
исходы не равновероятны: $P$(<<орел>>)$=0.6$, $P$(<<решка>>)$=0.4$.

\task Одна из сторон прямоугольника -- равномерно-распределенная случайная величина на интервале [6, 10]. 
Найти дисперсию площади прямоугольника, если другая его сторона равна 3 cм.

\task Всхожесть семян 36\%. Найти вероятность того, что более 22 семян из 100 прорастут 
(вычислить точное значение с 5-ю десятичными знаками).

\task Вычислить $(0.3, 0.6, 0.8)$-квантили из нормального распределения c параметрами $\mathcal{N}(5, 100)$.

\task Вычислить $(0.3, 0.6, 0.8)$-квантили для равномерного распределения на интервале [5, 15].

\subsection*{Вариант 2}
\setcounter{tasknum}{0}

\task Из полного набора 37 костей домино наудачу берутся 4 кости.
Найти вероятность того, что среди них будет хотя бы одна кость с 2 очками.

\task Монета подбрасывается 15 раз. Найти вероятность того, чточисло появлений герба нечетно.
 
\task В 8 ящиках размещают 16 шаров. Найти вероятность того,
что ни один ящик не пуст.

\task В круг вписан квадрат. Точка наудачу бросается в круг. Найти вероятность того, что она попадет в квадрат.

\task В урне 9 белых и 4 черных шара. 
Без возвращения извлекаются 3 шара. Известно, что среди них есть белый шар. 
Какова вероятность того, что другие два шара белые?

\task Полагая число очков игральной кости (возможные значения от 1 до 6) случайной величиной. 
Найти ее математическое ожидание и дисперсию.
 
\task Одна из сторон прямоугольника -- равномерно-распределенная случайная величина на интервале [10, 15]. 
Найти дисперсию площади прямоугольника, если другая его сторона равна 7 cм.

\task Всхожесть семян 30\%. Найти вероятность того, что более 30 семян из 100 прорастут 
(вычислить точное значение с 5-ю десятичными знаками).

\task Вычислить $(0.1, 0.2, 0.7)$-квантили для нормального распределения c параметрами $\mathcal{N}(5, 100)$. 

\task Вычислить $(0.1, 0.2, 0.7)$-квантили для равномерного распределения на интервале [0, 10]. 

 
 
 
\end{document}



