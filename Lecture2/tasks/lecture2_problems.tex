\documentclass[12pt, a4paper]{article}
\usepackage[utf8]{inputenc}
\usepackage[english,russian]{babel}
\usepackage{amssymb,amsfonts,amsmath,mathtext}
\usepackage{enumerate,indentfirst,tikz,pgf, float}

\topmargin=0mm

\footskip=6mm

\headheight=0pt

\headsep=0pt

\oddsidemargin -0.5cm

\evensidemargin -0.5cm

\textwidth 16.0cm

\textheight 25cm

\topmargin -1.5cm


% ----------- Счетчик задач ------------------------
\newcounter{tasknum}
\newcommand{\task}{\addtocounter{tasknum}{1}
\textbf{Задача \arabic{tasknum}.\,\,}}
%---------------------------------------------------


\begin{document}
\section*{Задания к лекции 2. Оценивание параметров.}

\subsection*{Вариант 1}


\task Найти эмпирическую функцию по данному расраспределению выборки ($x_i$ -- значения, $n_i$ -- частоты):
$$
\begin{array}{l}
x_i\qquad 4\; 7\; 8 \\
n_i\qquad 5\; 2\; 3
\end{array}
$$

\task Найти точечные оценки математического ожидания, дисперсии для данных \texttt{data\_p2\_var1}.

\task Вычислить точечные оценки математического ожидания (данные: \texttt{data\_p3\_var1}),  
 исходя из выборки из нормального распределения с параметрами $\mathcal{N}(3, 2)$:
$$
\begin{array}{l}
\sum\limits_i x_i, \\
\mbox{медиана}, \\
\max\limits_i x_i, \\
\min\limits_i x_i.
\end{array}
$$
Какая из перечисленных оценок является наиболее точной (учитывая, что распределение, для которого получена выборка известно).


\task Сгенерировав 1000 тестовых выборок из нормального распределения $\mathcal{N}(0,1)$ размером $3$ убедитесь, что
выражение для оценки дисперсии $\hat\sigma^2=\dfrac{1}{n-1}\sum\limits_i(x_i-\overline x)^2$, в отличие от
$\dfrac{1}{n}\sum\limits_i(x_i-\overline x)^2$ является более точной (правильнее несмещенной) оценкой $\sigma^2$.


\task Найти выборочную дисперсию по данному расраспределению выборки объема $n=100$:
$$
\begin{array}{l}
x_i\qquad  2502\; 2804\; 2903 \; 3028 \\
n_i\qquad 8\qquad 30\quad 60\quad\;\; 2
\end{array}
$$


\task Рассматриваются две оценки математического ожидания: среднее арифметическое значений и медиана. Выполнив имитационное 
моделирование -- создав 500 выборок размером 100 из равномерного распределения на интервале $[0, 1]$, дать оценку какая из них более эффективна 
(т.е. обладает меньшей дисперсией).
  

\task Оценкой максимального правдоподобия называется такая оценка, при которой наблюдаемая реализация наиболее вероятна.
Найти оценку максимального правдоподобия приживаемости саженцев, если из 100 саженцев прижилось 60.


\task Построить интервальные оценки дисперсии и математического ожидания для выборки \texttt{bigdataset\_var1}, 
полученной из нормального распределения.


\task Найти минимальный объем выборки, при котором с надежностью $0.975$ 
точность оценки математического ожидания $a$ генеральной совокупности по 
выборочной средней равна  $\delta = 0.2$, 
если известно среднее квадратическое отклонение $\sigma = 1.1$ 
нормально распределенной генеральной совокупности.


\task В 360 испытаниях, в каждом из которых вероятность появления события одинакова и неизвестна, 
событие $A$ появилось 270 раз. Найти доверительный интервал, покрывающий неизвестную вероятность $P$ с надежностью 0.95.

\task Исследуется распределение безразмерного параметра некоторого биологического объекта (например, 
отношение длины и ширины листовой пластинки, длины и ширины семени и пр.). Возможны два подхода к вычислению среднего
отношения: $m_1=\dfrac{\sum\limits_i l_i}{\sum\limits_j h_j}$ и $m_2=\mbox{median}\left\{\dfrac{l_i}{h_i}\right\}$, $i=\overline{1\ldots n}$.
На основе статистического моделирования с выборками из нормального распределения установить какой из этих подходов
приводит к более эффективной оценке отношения <<длины>> ($l$) и <<высоты>> ($h$) объекта.


\clearpage
\newpage
\section*{Задания к лекции 2. Оценивание параметров.}
\subsection*{Вариант 2}
\setcounter{tasknum}{0}
\setcounter{figure}{0}

\task Найти эмпирическую функцию по данному расраспределению выборки ($x_i$ -- значения, $n_i$ -- частоты):
$$
\begin{array}{l}
x_i\qquad 2\; 5 \; 7\; 8 \\
n_i\qquad 1\; 3\; 2\; 4
\end{array}
$$

\task Найти точечные оценки математического ожидания, дисперсии для данных \texttt{data\_p2\_var2}.
 
 
\task Вычислить точечные оценки математического ожидания (данные: \texttt{data\_p3\_var2}),  
 исходя из выборки из нормального распределения с параметрами $\mathcal{N}(5, 3)$:
$$
\begin{array}{l}
\sum\limits_i x_i, \\
\mbox{медиана}, \\
\max\limits_i x_i, \\
\min\limits_i x_i.
\end{array}
$$
Какая из перечисленных оценок является наиболее точной (учитывая, что распределение, для которого получена выборка известно).
 
\task Сгенерировав 1000 тестовых выборок из нормального распределения $\mathcal{N}(0,1)$ размером $5$ убедитесь, что
выражение для оценки дисперсии $\hat\sigma^2=\dfrac{1}{n-1}\sum\limits_i(x_i-\overline x)^2$, где 
$\overline x=\dfrac{1}{n}\sum\limits_i x_i$ в отличие от
$\dfrac{1}{n}\sum\limits_i(x_i-\overline x)^2$ является более точной (правильнее -- несмещенной) оценкой $\sigma^2$. 
 
\task Найти выборочную дисперсию по данному расраспределению выборки объема $n=100$:
$$
\begin{array}{l}
x_i\qquad  340\; 360\;375\; 380   \\
n_i\qquad  20\;\;\; 50\;\;\; 18\;\;\; 12
\end{array}
$$
 
\task Рассматриваются две оценки математического ожидания: среднее арифметическое значений и медиана. Выполнив имитационное 
моделирование -- создав 400 выборок размером 120 из нормального распределения $\mathcal{N}(1, 1)$, 
дать оценку какая из них менее эффективна (т.е. обладает большей дисперсией).

\task Оценкой максимального правдоподобия называется такая оценка, при которой наблюдаемая реализация наиболее вероятна.
Найти оценку максимального правдоподобия приживаемости саженцев, если из 100 саженцев прижилось 30.
 
\task Построить интервальные оценки дисперсии и математического ожидания для выборки \texttt{bigdataset\_var2},
полученной из нормального распределения.

\task Найти минимальный объем выборки, при котором с надежностью $0.975$ 
точность оценки математического ожидания $a$ генеральной совокупности по 
выборочной средней равна  $\delta = 0.3$, 
если известно среднее квадратическое отклонение $\sigma = 1.2$ 
нормально распределенной генеральной совокупности.


\task Произведено 300 испытаний по схеме Бернулли. Событие $A$ появилось в 250 испытаниях.
Найти доверительный интервал, покрывающий неизвестную вероятность $P(A)$ c надежностью 0.95.

\task Исследуется распределение безразмерного параметра некоторого биологического объекта (например, 
отношение длины и ширины листовой пластинки, длины и ширины семени и пр.). Возможны два подхода к вычислению среднего
отношения: $m_1=\dfrac{\sum\limits_i l_i}{\sum\limits_j h_j}$ и $m_2=\dfrac{1}{n}\sum\limits_{i=1}^n\dfrac{l_i}{h_i}$.
На основе статистического моделирования с выборками из нормального распределения установить какой из этих подходов
приводит к более эффективной оценке отношения <<длины>> ($l$) и <<высоты>> ($h$) объекта. 


 
\end{document}



