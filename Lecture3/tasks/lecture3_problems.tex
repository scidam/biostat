\documentclass[12pt, a4paper]{article}
\usepackage[utf8]{inputenc}
\usepackage[english,russian]{babel}
\usepackage{amssymb,amsfonts,amsmath,mathtext}
\usepackage{enumerate,indentfirst,tikz,pgf, float}

\topmargin=0mm

\footskip=6mm

\headheight=0pt

\headsep=0pt

\oddsidemargin -0.5cm

\evensidemargin -0.5cm

\textwidth 16.0cm

\textheight 25cm

\topmargin -1.5cm


% ----------- Счетчик задач ------------------------
\newcounter{tasknum}
\newcommand{\task}{\addtocounter{tasknum}{1}
\textbf{Задача \arabic{tasknum}.\,\,}}
%---------------------------------------------------


\begin{document}
\section*{Задания к лекции 3.  Анализ нечисловых данных.}

\subsection*{Вариант 1}

\task Вычислить меры сходства Серенсена и Жаккара для сравниваемых наборов 
 ['Г', 'А', 'Б', 'И', 'Т', 'У', 'С'] и  ['О', 'Д', 'У', 'В', 'А', 'Н', 'Ч', 'И', 'К']
 без учета порядка входящих букв.

\task Сравниваются флористические списки $A$ и $B$. Если к списку $A$ добавить несколько новых, не содержащихся в $B$ видов, 
как поведет себя мера Жаккара (увеличится, уменьшится, останется неизменной)?


\task Показать, что в параметризации Б.И. Семкина $K_{0;-1}$ -- совпадает с выражением для коэффициента Серенсена.
$$
\begin{array}{l} K_{\tau;\eta} =\left( \dfrac{K_{\tau}^{\eta}(A,B)+K_{\tau}^{\eta}(B,A)}{2}\right)^{1/\eta},
\\ K_{\tau}(A,B) = \dfrac{|A\cap B|}{(1+\tau)|A|-\tau|A\cap B|},\\
K_{\tau}(B,A) = \dfrac{|A\cap B|}{(1+\tau)|B|-\tau|A\cap B|} \\ -1<\tau<\infty, -\infty<\eta<\infty \end{array}
$$

\task Зависит ли скорость изменения коэффициента Жаккара при увеличении числа общих элементов двух сравниваемых множеств,
если число элементов объединения этих множеств в процессе изменения остается постоянной.

\task Даны два мультимножества, состоящие из символов:
$$
\begin{array}{l}
A = [1,1,2,2,2,4,7,8,9,p,p]
B = [4,5,2,2,2,1,4,4,4,4,q,q]
\end{array}
$$
Вычислить меры Жаккара, Дайса и Кульчинского.






\task Чему равна вероятность наблюдать таблицу сопряженности $2\times 2$ следующего вида:
\begin{table}[H]
\centering
\begin{tabular}{l|l|}
$7$ & $50$  \\\hline
$45$ & $38$ \\\hline
\end{tabular}
\end{table}
Привести точное выражение для вероятности и вычислить приближенное значение 
(для вычислений можно использовать среду статистическго анализа R или Python).


\task Для таблицы сопряженности из предыдущей задачи применить критерий $\chi^2$ (при уровне значимости $0.03$)
с целью исследования зависимости признаков
(для вычислений можно использовать среду статистическго анализа R или Python).


\clearpage
\newpage
\section*{Задания к лекции 3. Анализ нечисловых данных.}
\subsection*{Вариант 2}
\setcounter{tasknum}{0}
\setcounter{figure}{0}

\task Вычислить меры сходства Серенсена и Жаккара для сравниваемых наборов 
 ['М', 'О', 'Р', 'К', 'О', 'В', 'К', 'А'] и  ['М', 'А', 'Р', 'Т', 'Ы', 'Ш', 'К', 'А'] с учетом порядка входящих букв.
 
 \task Сравниваются флористические списки $A$ и $B$. 
 Если к списку $A$ добавить несколько новых, не содержащихся в списке $B$ видов, 
как поведет себя мера Серенсена-Дайса (увеличится, уменьшится, останется неизменной)?

\task Показать, что в параметризации Б.И. Семкина $K_{1;-1}$ -- совпадает с выражением для коэффициента Жаккара.
$$
\begin{array}{l} K_{\tau;\eta} =\left( \dfrac{K_{\tau}^{\eta}(A,B)+K_{\tau}^{\eta}(B,A)}{2}\right)^{1/\eta},
\\ K_{\tau}(A,B) = \dfrac{|A\cap B|}{(1+\tau)|A|-\tau|A\cap B|},\\
K_{\tau}(B,A) = \dfrac{|A\cap B|}{(1+\tau)|B|-\tau|A\cap B|} \\ -1<\tau<\infty, -\infty<\eta<\infty \end{array}
$$


\task Зависит ли скорость изменения коэффициента Дайса при увеличении числа общих элементов
двух сравниваемых множеств, если число элементов объединения этих множеств
в процессе изменения остается постоянной.


\task Рассматривая две строки как мультимножества, состоящие из символов:
$$
\begin{array}{l}
A = 'jaccard1901',
B = 'dice1948',
\end{array}
$$
вычислить меры Жаккара, Дайса и Кульчинского.




\task Чему равна вероятность наблюдать таблицу сопряженности $2\times 2$ следующего вида:
\begin{table}[H]
\centering
\begin{tabular}{l|l|}
$25$ & $35$  \\\hline
$4$ & $18$ \\\hline
\end{tabular}
\end{table}
Привести точное выражение для вероятности и вычислить приближенное значение 
(для вычислений можно использовать среду статистическго анализа R или Python).

\task Для таблицы сопряженности из предыдущей задачи применить критерий $\chi^2$ (при уровне значимости $0.02$)
с целью исследования зависимости признаков
(для вычислений можно использовать среду статистическго анализа R или Python).
 

 
 
\end{document}



