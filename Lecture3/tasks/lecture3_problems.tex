\documentclass[12pt, a4paper]{article}
\usepackage[utf8]{inputenc}
\usepackage[english,russian]{babel}
\usepackage{amssymb,amsfonts,amsmath,mathtext}
\usepackage{enumerate,indentfirst,tikz,pgf, float}

\topmargin=0mm

\footskip=6mm

\headheight=0pt

\headsep=0pt

\oddsidemargin -0.5cm

\evensidemargin -0.5cm

\textwidth 16.0cm

\textheight 25cm

\topmargin -1.5cm


% ----------- Счетчик задач ------------------------
\newcounter{tasknum}
\newcommand{\task}{\addtocounter{tasknum}{1}
\textbf{Задача \arabic{tasknum}.\,\,}}
%---------------------------------------------------


\begin{document}
\section*{Задания к лекции 3.  Анализ нечисловых данных.}

\subsection*{Вариант 1}

\task Вычислить меры сходства Серенсена и Жаккара для сравниваемых наборов 
 ['Г', 'А', 'Б', 'И', 'Т', 'У', 'С'] и  ['О', 'Д', 'У', 'В', 'А', 'Н', 'Ч', 'И', 'К']
 без учета порядка входящих букв.

\task Сравниваются флористические списки $A$ и $B$. Если к списку $A$ добавить несколько новых, не содержащихся в $B$ видов, 
как поведет себя мера Жаккара (увеличится, уменьшится, останется неизменной)?




\clearpage
\newpage
\section*{Задания к лекции 3. Анализ нечисловых данных.}
\subsection*{Вариант 2}
\setcounter{tasknum}{0}
\setcounter{figure}{0}

\task Вычислить меры сходства Серенсена и Жаккара для сравниваемых наборов 
 ['М', 'О', 'Р', 'К', 'О', 'В', 'К', 'А'] и  ['М', 'А', 'Р', 'Т', 'Ы', 'Ш', 'К', 'А'] с учетом порядка входящих букв.
 
 \task Сравниваются флористические списки $A$ и $B$. 
 Если к списку $A$ добавить несколько новых, не содержащихся в списке $B$ видов, 
как поведет себя мера Серенсена-Дайса (увеличится, уменьшится, останется неизменной)?


 
 
\end{document}



