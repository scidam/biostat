\documentclass[12pt, a4paper]{article}
\usepackage[utf8]{inputenc}
\usepackage[english,russian]{babel}
\usepackage{amssymb,amsfonts,amsmath,mathtext}
\usepackage{enumerate,indentfirst,tikz,pgf, float}

\usepackage{tikz}
\usetikzlibrary{arrows,%
                shapes,positioning}

\topmargin=0mm

\footskip=6mm

\headheight=0pt

\headsep=0pt

\oddsidemargin -0.5cm

\evensidemargin -0.5cm

\textwidth 16.0cm

\textheight 25cm

\topmargin -1.5cm


% ----------- Счетчик задач ------------------------
\newcounter{tasknum}
\newcommand{\task}{\addtocounter{tasknum}{1}
\textbf{Задача \arabic{tasknum}.\,\,}}
%---------------------------------------------------


\begin{document}
\section*{Задания к лекции 3.  Анализ нечисловых данных.}

\subsection*{Вариант 1}

\task Вычислить меры сходства Серенсена и Жаккара для сравниваемых наборов 
 ['Г', 'А', 'Б', 'И', 'Т', 'У', 'С'] и  ['О', 'Д', 'У', 'В', 'А', 'Н', 'Ч', 'И', 'К']
 без учета порядка входящих букв.

\task Сравниваются флористические списки $A$ и $B$. Если к списку $A$ добавить несколько новых, не содержащихся в $B$ видов, 
как поведет себя мера Жаккара (увеличится, уменьшится, останется неизменной)?


\task Показать, что в параметризации Б.И. Семкина $K_{0;-1}$ -- совпадает с выражением для коэффициента Серенсена.
$$
\begin{array}{l} K_{\tau;\eta} =\left( \dfrac{K_{\tau}^{\eta}(A,B)+K_{\tau}^{\eta}(B,A)}{2}\right)^{1/\eta},
\\ K_{\tau}(A,B) = \dfrac{|A\cap B|}{(1+\tau)|A|-\tau|A\cap B|},\\
K_{\tau}(B,A) = \dfrac{|A\cap B|}{(1+\tau)|B|-\tau|A\cap B|} \\ -1<\tau<\infty, -\infty<\eta<\infty \end{array}
$$

\task Зависит ли скорость изменения коэффициента Жаккара при увеличении числа общих элементов двух сравниваемых множеств,
если число элементов объединения этих множеств в процессе изменения остается постоянной.

\task Даны два мультимножества, состоящие из символов:
$$
\begin{array}{l}
A = [1,1,2,2,2,4,7,8,9,p,p]
B = [4,5,2,2,2,1,4,4,4,4,q,q]
\end{array}
$$
Вычислить меры Жаккара, Дайса и Кульчинского.


\task Вычислить меру Дайса для двух списков (с повторениями) видов \texttt{lec3_measures_var1.dat}.
\begin{tabular}{ll}
Caldesia reniformis;&Caldesia reniformis\\
Caldesia reniformis;&Strobilanthes isophyllus\\
Strobilanthes isophyllus;&Onychium japonicum\\
Pseuderanthemum atropurpureum;&Agave filifera\\
Caldesia reniformis;&Allium spirale\\
\end{tabular} 
\ldots


\task Вы сравниваете флористические списки. Матрица мер сходства Жаккара успешно посчитана до вас, но содержит ошибки.
Попробуйте определить при сравнении каких именно списков (они нумеруются по строкам\столбцам матрицы)
была допущена ошибка. Матрица дана в файле \texttt{lec3_bigmat_var1.dat}


\task Исследуется эффективность действия удобрения на рост растений. Для этого 100 тестовых образцов 
были разделены в пропорции 2:3 на тестовые (не подвергавшиеся обработки удобрением) и остальные, которые
подлежали обработке. Через месяц был произведен  учет растений, в результате которого были подсчитаны растения
показавшие 10 сантиметровый прирост.
\begin{tabular}{l|l|l|l}
&  не удобрено & удобрено & $\sum$ \\ \hline
прирост>$10$ см & $10$ & $43$ & $53$ \\\hline
прирост<$10$ см & $30$ & $17$ & $47$ \\\hline
$\sum$ & $40$ & $60$ & $100$
\end{tabular}
Можно ли сказать, что данные результаты не следствие случая, а действия удобрения?



\task Чему равна вероятность наблюдать таблицу сопряженности $2\times 2$ следующего вида:
\begin{table}[H]
\centering
\begin{tabular}{l|l|}
$7$ & $50$  \\\hline
$45$ & $38$ \\\hline
\end{tabular}
\end{table}
Привести точное выражение для вероятности и вычислить приближенное значение 
(для вычислений можно использовать среду статистическго анализа R или Python).



\task Для таблицы сопряженности из предыдущей задачи применить критерий $\chi^2$ (при уровне значимости $0.03$)
с целью исследования зависимости признаков
(для вычислений можно использовать среду статистическго анализа R или Python).


\task Граф задан матрицей инциденций. Найти его матрицу смежности.
$$
\begin{array}{lllllll}
1 & 0 & 0 & 0 & 0 & 0 & 1 \\
0 & 1 & 0 & 0 & 0 & 1 & 1 \\
1 & 0 & 0 & 1 & 0 & 0 & 0 \\
0 & 1 & 1 & 1 & 0 & 0 & 0 \\
0 & 1 & 1 & 0 & 1 & 0 & 0 
\end{array}
$$

\task В результате исследования 100 пробных площадей были вычислены их попарные меры сходства. 
Сходство между пробными площадями считали существенным, если между $i$-й и $j$-й площадками
вычисленный коэффициент был больше заданного порога. По результатам анализа всех комбинаций была
построена матрица смежности: если сходство между $i$ и $j$ было существенным, то соответствующий $(i,j)$-элемент
матрицы выбирался равным 1, в противном случае 0. Найти все совокупности сходных групп пробных площадей, являющиеся
компонентами связности графа заданного построенной матрицей смежности. Матрица смежности дана в файле: 
\texttt{lec3_datamat_var1.dat}



\task Граф задан матрицей смежности. Найти его матрицу инциденций.
$$
\begin{array}{llll}
0 & 1 & 1 & 1\\
1 & 0 & 1 & 1\\
1 & 1 & 0 & 1\\
1 & 1 & 1 & 0\\
\end{array}
$$



\task Чему равна сумма элементов матрицы инциденций для полного графа с числом вершин $n$?



\task Чему равна вероятность реализации графа
\begin{tikzpicture}[node distance   = 2 cm]
  \useasboundingbox (-1,-1) rectangle (11,11); 
  \tikzset{VertexStyle/.style = {shape          = circle,
                                 ball color     = orange,
                                 text           = black,
                                 inner sep      = 2pt,
                                 outer sep      = 0pt,
                                 minimum size   = 24 pt}}
  \tikzset{EdgeStyle/.style   = {thick,
                                 double          = orange,
                                 double distance = 1pt}}
  \tikzset{LabelStyle/.style =   {draw,
                                  fill           = yellow,
                                  text           = red}}
     \node[VertexStyle](A){A};
     \node[VertexStyle,right=of A](B){B};
     \node[VertexStyle,right=of B](C){C};
     \node[VertexStyle,above= 2 cm of B](D){D};
     \node[VertexStyle,right=of D](E){E};     
     \draw[EdgeStyle](B) to node[LabelStyle]{1} (D) ;
     \tikzset{EdgeStyle/.append style = {bend left}}
     \draw[EdgeStyle](A) to node[LabelStyle]{2} (B);
     \draw[EdgeStyle](B) to node[LabelStyle]{3} (A);
     \draw[EdgeStyle](B) to node[LabelStyle]{4} (C);
     \draw[EdgeStyle](C) to node[LabelStyle]{5} (B);
     \draw[EdgeStyle](A) to node[LabelStyle]{6} (D);
     \draw[EdgeStyle](D) to node[LabelStyle]{7} (C);
     \draw[EdgeStyle](B) to node[LabelStyle]{8} (E);
     \draw[EdgeStyle](A) to node[LabelStyle]{9} (E);
  \end{tikzpicture}
Если известно, что у графа обязательно должно быть 9 ребер. 


\clearpage
\newpage
\section*{Задания к лекции 3. Анализ нечисловых данных.}
\subsection*{Вариант 2}
\setcounter{tasknum}{0}
\setcounter{figure}{0}

\task Вычислить меры сходства Серенсена и Жаккара для сравниваемых наборов 
 ['М', 'О', 'Р', 'К', 'О', 'В', 'К', 'А'] и  ['М', 'А', 'Р', 'Т', 'Ы', 'Ш', 'К', 'А'] с учетом порядка входящих букв.
 
 \task Сравниваются флористические списки $A$ и $B$. 
 Если к списку $A$ добавить несколько новых, не содержащихся в списке $B$ видов, 
как поведет себя мера Серенсена-Дайса (увеличится, уменьшится, останется неизменной)?

\task Показать, что в параметризации Б.И. Семкина $K_{1;-1}$ -- совпадает с выражением для коэффициента Жаккара.
$$
\begin{array}{l} K_{\tau;\eta} =\left( \dfrac{K_{\tau}^{\eta}(A,B)+K_{\tau}^{\eta}(B,A)}{2}\right)^{1/\eta},
\\ K_{\tau}(A,B) = \dfrac{|A\cap B|}{(1+\tau)|A|-\tau|A\cap B|},\\
K_{\tau}(B,A) = \dfrac{|A\cap B|}{(1+\tau)|B|-\tau|A\cap B|} \\ -1<\tau<\infty, -\infty<\eta<\infty \end{array}
$$


\task Зависит ли скорость изменения коэффициента Дайса при увеличении числа общих элементов
двух сравниваемых множеств, если число элементов объединения этих множеств
в процессе изменения остается постоянной.


\task Рассматривая две строки как мультимножества, состоящие из символов:
$$
\begin{array}{l}
A = 'jaccard1901',
B = 'dice1948',
\end{array}
$$
вычислить меры Жаккара, Дайса и Кульчинского.

\task Вычислить меру Жаккара для двух списков (с повторениями) видов \texttt{lec3_measures_var2.dat}.
\begin{tabular}{ll}
Agave stricta; &Lithops aucampiae\\
Aloe aristata; &Pseuderanthemum atropurpureum\\
Aloe aristata; &Allium schoenoprasum\\
Plumeria rubra; &Hosta rectifolia\\
Allium spirale;&Allium schoenoprasum\\
\end{tabular} 
\ldots

\task Вы сравниваете флористические списки. Матрица мер сходства успешно посчитана до вас, но содержит ошибки.
Попробуйте определить при сравнении каких именно списков (они нумеруются по строкам\столбцам матрицы)
была допущена ошибка. Матрица дана в файле \texttt{lec3_bigmat_var2.dat}


\task Исследуется эффективность действия удобрения на рост растений. Для этого 100 тестовых образцов 
были разделены в пропорции 1:3 на тестовые (не подвергавшиеся обработки удобрением) и остальные, которые
подлежали обработке. Через месяц был произведен  учет растений, в результате которого были подсчитаны растения
показавшие 10 сантиметровый прирост.
\begin{tabular}{l|l|l|l}
&  не удобрено & удобрено & $\sum$ \\ \hline
прирост>$10$ см & $6$ & $18$ & $24$ \\\hline
прирост<$10$ см & $19$ & $57$ & $76$ \\\hline
$\sum$ & $25$ & $75$ & $100$
\end{tabular}
Можно ли сказать, что данные результаты не следствие случая, а действия удобрения?


\task Чему равна вероятность наблюдать таблицу сопряженности $2\times 2$ следующего вида:
\begin{table}[H]
\centering
\begin{tabular}{l|l|}
$25$ & $35$  \\\hline
$4$ & $18$ \\\hline
\end{tabular}
\end{table}
Привести точное выражение для вероятности и вычислить приближенное значение 
(для вычислений можно использовать среду статистическго анализа R или Python).

\task Для таблицы сопряженности из предыдущей задачи применить критерий $\chi^2$ (при уровне значимости $0.02$)
с целью исследования зависимости признаков
(для вычислений можно использовать среду статистическго анализа R или Python).
 

\task Граф задан матрицей инциденций. Найти его матрицу смежности.
$$
\begin{array}{lllllll}
0 & 1 & 1 & 1 & 0 & 0 & 0 \\
0 & 0 & 0 & 0 & 1 & 1 & 0 \\
1 & 0 & 1 & 0 & 1 & 0 & 0 \\
0 & 0 & 1 & 1 & 1 & 0 & 0 \\
1 & 1 & 1 & 0 & 1 & 0 & 0 
\end{array}
$$

\task В результате исследования 100 пробных площадей были вычислены их попарные меры сходства. 
Сходство между пробными площадями считали существенным, если между $i$-й и $j$-й площадками
вычисленный коэффициент был больше заданного порога. По результатам анализа всех комбинаций была
построена матрица смежности: если сходство между $i$ и $j$ было существенным, то соответствующий $(i,j)$-элемент
матрицы выбирался равным 1, в противном случае 0. Найти все совокупности сходных групп пробных площадей, являющиеся
компонентами связности графа заданного построенной матрицей смежности. Матрица смежности дана в файле: 
\texttt{lec3_datamat_var2.dat}


\task Граф задан матрицей смежности. Найти его матрицу инциденций.
$$
\begin{array}{lllll}
0 & 0 & 1 & 1 & 1\\
0 & 1 & 0 & 0 & 0\\
0 & 0 & 0 & 0 & 1\\
0 & 1 & 1 & 0 & 0\\
1 & 0 & 1 & 1 & 0\\
\end{array}
$$


\task Чему равна сумма элементов матрицы смежности для полного графа с числом вершин $n$? 
 
\task Чему равна вероятность реализации графа
\begin{tikzpicture}[node distance   = 2 cm]
  \useasboundingbox (-1,-1) rectangle (11,11); 
  \tikzset{VertexStyle/.style = {shape          = circle,
                                 ball color     = orange,
                                 text           = black,
                                 inner sep      = 2pt,
                                 outer sep      = 0pt,
                                 minimum size   = 24 pt}}
  \tikzset{EdgeStyle/.style   = {thick,
                                 double          = orange,
                                 double distance = 1pt}}
  \tikzset{LabelStyle/.style =   {draw,
                                  fill           = yellow,
                                  text           = red}}
     \node[VertexStyle](A){A};
     \node[VertexStyle,right=of A](B){B};
     \node[VertexStyle,right=of B](C){C};
     \node[VertexStyle,above= 2 cm of B](D){D};     
     \draw[EdgeStyle](B) to node[LabelStyle]{1} (D) ;
     \tikzset{EdgeStyle/.append style = {bend left}}
     \draw[EdgeStyle](A) to node[LabelStyle]{2} (B);
     \draw[EdgeStyle](B) to node[LabelStyle]{3} (A);
     \draw[EdgeStyle](B) to node[LabelStyle]{4} (C);
     \draw[EdgeStyle](C) to node[LabelStyle]{5} (B);
     \draw[EdgeStyle](A) to node[LabelStyle]{6} (D);
     \draw[EdgeStyle](D) to node[LabelStyle]{7} (C);
  \end{tikzpicture}
Если известно, что у графа обязательно должно быть 7 ребер. 
 
 
\end{document}



